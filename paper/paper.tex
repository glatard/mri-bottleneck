\documentclass[conference,draft]{IEEEtran}

\usepackage{cite}
\usepackage{xcolor}
\usepackage{hyperref}
\hypersetup{colorlinks=true,linkcolor=black,citecolor=blue,filecolor=black,urlcolor=blue}
\usepackage{graphicx}
\graphicspath{{figures/}}

\newcommand{\TG}[1]{\color{red}\textsc{From Tristan: }#1\color{black}}
\newcommand{\MD}[1]{\color{magenta}\textsc{From Tristan: }#1\color{black}}
\newcommand{\HL}[1]{\hl{#1}}

\title{Surveying the Performance Bottleneck of Neuroimaging Pre-Processing}

\author{Mathieu Dugr\'e, Tristan Glatard}

\begin{document}
\maketitle

\begin{abstract}
	% TODO
																											
\end{abstract}


\section{Introduction}
% TODO

% MRI preprocessing

% Motivate the need to profile the bottleneck of applications.
% * Major part of neuroimaging study
% * Faster pre-processing would enable larger study and more clinical use.
% Identifying the major bottleneck will let us know what to optimize down the line. (compute, compression, etc.)
% Reduced precision + Compression ?

\section{Background}
% TODO
\subsection{MRI Pre-Processing}
% MRI modalities
% Pre-processing tools
% Pre-processing step (fMRIPrep figure, + something for DIPY)


\subsection{Profiling (VTune)}
% Either profiling in general, or VTune in particular.

% VTune
% Lightweight profiler
% Info on function and module runtime

\subsection{Reduced Precision}
% Quickly motivate the need for this paper
% Summarize the advantages and disavantages of using reduced precison.
% Talk about the potential benefits it could have to the field of neurorimaging,
% but that profiling the application hotspot is required to perform it.

\subsection{Compression}
% How it can help with intermediate file.
% Link to reduced precision, which could enable a sort of compression by of it self.

\section{Methods}
\subsection{Infrastructure}
% ??? Should we include that ???
% Doesn't really mattter for this paper, I think.
% But doesn't harm either. For example, the performance could vary depending on architecture.

\subsection{Dataset}
% TODO
% size
% distribution: age, sex, race?, etc.
% Look into CoRR and HPC datasets.
% Need to have quality, yet not prestine, data with subject info for at least age and sex.

\subsection{Pipelines}
% TODO 
% Motivation for those pipelines:

% MRI modalities
% fMRIprep (full) + sub-pipelines (sMRI + fMRI)
% DIPY pipelines (dMRI)

% Use case from papers
% LivingPark pipelines (sMRI)

\section{Results}
% TODO
% 2-step approach:
% 1. Which pipeline constitute the long execution time in preprocessing?
% 2. For those, which functions take the most time to execute? (hotspot)

\subsection{Makespan}
% Identify which tools take the longest to execute.

% ###    FIGURE    ###
%  Bar chart with the total makespan of the pipeline execution (mean + std)
% ### END FIGURE ####
% This will let us decided which pipeline to investigate more.

\subsection{Hotspot}
% (1-2 pages)
% TopK (approx. 10) longest makepsan application. The rest can go in supplemental.
% ###    FIGURE    ###
% (Large pipeline)
% fMRIPrep
% sMRIPrep
% DIPY
% 
% (Pipeline components)
% FreeSurfer
% ANTs ??
% SPM DARTEL
% ### END FIGURE ####

% Supplementary figure for the sub-components of large pipelines.

\section{Discussion}
% TODO
% Principal Bottlenecks

% Limitation
% Can only perform profiling of open-source pipelines
% Or pipelines with debug-info.

% Opportunity for optimisation (future work)
% Reduced precision for: Compute or Compression

\section{Conclusion}
% TODO

\section*{Acknowledgement}
% TODO

\section*{Data Availability}
% TODO

\section*{Conflict of Interests}
The authors report no conflict of interests.

\section*{Supplementary Figures}
% Pipelines' hotspot with bottomK makespan

\bibliographystyle{IEEEtran}
\bibliography{IEEEabrv, paper.bib}


\end{document}
